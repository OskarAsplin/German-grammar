\documentclass{article}
\usepackage[lmargin=25mm,rmargin=25mm,tmargin=20mm,bmargin=30mm]{geometry}
\usepackage[utf8]{inputenc}
\usepackage{tabu, float, bm}

\newcommand{\specialcell}[3][c]{%
  \begin{tabular}[#1]{@{}#2@{}}#3\end{tabular}}

\title{German grammar}
\author{Oskar Asplin}
\date{March 2018}

\begin{document}

\maketitle

\section{Grammar tables}

\begin{table}[H]
    \centering
    \begin{tabu} to 1.0\textwidth { |X[c]|X[c]|X[c]|X[c]|X[c]| } 
        \hline
        \multicolumn{5}{|c|}{\textbf{Definite Article}} \\
        \hline
         & Masculine & Neuter & Feminine & Plural \\
        \hline
        Nominative & Der & Das & \multicolumn{2}{|c|}{Die} \\ \cline{1-2}
        Accusative & Den & & \multicolumn{2}{|c|}{} \\
        \hline
        Dative & \multicolumn{2}{|c|}{Dem} & & Den\\
        \cline{1-3} \cline{5-5}
        Genitive & \multicolumn{2}{|c|}{Des} & \multicolumn{2}{|l|}{\hspace{1.3cm}Der} \\
        \hline
    \end{tabu}
    \caption{Same for diese}
    \label{tab:definitiv_article}
\end{table}

\begin{table}[H]
    \centering
    \begin{tabu} to 1.0\textwidth { |X[c]|X[c]|X[c]|X[c]|X[c]| } 
        \hline
        \multicolumn{5}{|c|}{\textbf{Strong Adjective}} \\
        \hline
         & Masculine & Neuter & Feminine & Plural \\
        \hline
        Nominative & -er & -es & \multicolumn{2}{|c|}{-e} \\ \cline{1-2}
        Accusative & -en & & \multicolumn{2}{|c|}{} \\
        \hline
        Dative & \multicolumn{2}{|c|}{-em} & & -en \\ \cline{1-3} \cline{5-5}
        Genitive & \multicolumn{2}{|c|}{-en} & \multicolumn{2}{|l|}{\hspace{1.3cm}-er} \\
        \hline
    \end{tabu}
    \caption{When there is no article or after indefinite article which is not conjugated}
    \label{tab:strong_adjetive}
\end{table}

\begin{table}[H]
    \centering
    \begin{tabu} to 1.0\textwidth { |X[c]|X[c]|X[c]|X[c]|X[c]| } 
        \hline
        \multicolumn{5}{|c|}{\textbf{Weak Adjective}} \\
        \hline
         & Masculine & Neuter & Feminine & Plural \\
        \hline
        Nominative & \multicolumn{3}{|c|}{-e} & \\ \cline{1-2}
        Accusative &  & \multicolumn{2}{|c|}{} & \\
        \cline{1-1} \cline{3-4}
        Dative & \multicolumn{4}{|c|}{-en} \\ \cline{1-1}
        Genitive & \multicolumn{4}{|c|}{} \\
        \hline
    \end{tabu}
    \caption{After definite article or indefinite article with conjugation}
    \label{tab:weak_adjetive}
\end{table}

\begin{table}[H]
    \centering
    \begin{tabu} to 1.0\textwidth { |X[c]|X[c]|X[c]|X[c]|X[c]| } 
        \hline
        \multicolumn{5}{|c|}{\textbf{Indefinite Article}} \\
        \hline
         & Masculine & Neuter & Feminine & Plural \\
        \hline
        Nominative & \multicolumn{2}{|c|}{Ein} & Eine & (Meine) \\ \cline{1-2} \cline{5-5}
        Accusative & Einen & & & (Meine) \\
        \hline
        Dative & \multicolumn{2}{|c|}{Einem} & Einer & (Meinen) \\
        \cline{1-3} \cline{5-5}
        Genitive & \multicolumn{2}{|c|}{Eines} & & - \\
        \hline
    \end{tabu}
    \caption{Same for Mein/Dein/sein/ihr/sein/unser/euer/ihr/ihr}
    \label{tab:ein}
\end{table}

\begin{table}[H]
    \centering
    \begin{tabu} to 1.0\textwidth { |X[c]|X[c]|X[c]|X[c]|X[c]|X[c]| } 
        \hline
        \multicolumn{6}{|c|}{\textbf{German pronouns - Singular}} \\
        \hline
         & I (me) & You - informal & He (him) & She (her) & It \\
        \hline
        Nominative & Ich & Du & Er & Sie & Es \\ \cline{1-4}
        Accusative & Mich & Dich & Ihn & & \\
        \hline
        Dative & Mir & Dir & Ihm & Ihr & Ihm \\
        \hline
        (Possesive) & Mein & Dein & Sein & Ihr & Sein \\
        \hline
    \end{tabu}
    \caption{German pronouns - Singular}
    \label{tab:pron_sing}
\end{table}

\begin{table}[H]
    \centering
    \begin{tabu} to 1.0\textwidth { |X[c]|X[c]|X[c]|X[c]| } 
        \hline
        \multicolumn{4}{|c|}{\textbf{German pronouns - Plural}} \\
        \hline
         & We (us) & You - informal & They \\
        \hline
        Nominative & Wir & Ihr & Sie \\ 
        \cline{1-3}
        Accusative & Uns & Euch & \\
        \cline{1-1} \cline{4-4}
        Dative & & & Ihnen \\
        \hline
        (Possesive) & Unser & Euer & Ihr \\
        \hline
    \end{tabu}
    \caption{German pronouns - Plural}
    \label{tab:pron_plur}
\end{table}

\section{Rules}

\begin{table}[H]
    \centering
    \begin{tabu} to 0.7\textwidth { | X[c] | X[c] | }
        \hline
        Always accusative & durch, für, gegen, ohne, um \\
        \hline
        Always dative & aus, bei, mit, nach, von, zu \\
        \hline
    \end{tabu}
    \caption{Accusative and dative prepositions}
    \label{tab:acc_or_dat}
\end{table}

\begin{table}[H]
    \centering
    \begin{tabu} to 0.7\textwidth { | X[c] | X[c] | }
        \hline
        \multicolumn{2}{|c|}{an, auf, hinter, in, neben, über, unter, vor, zwischen} \\
        \hline
        Accusative & Movement to the place \\
        \hline
        Dative & Movement at the place \\
        \hline
    \end{tabu}
    \caption{Alternating prepositions}
    \label{tab:alt_prep}
\end{table}

\begin{table}[H]
    \centering
    \begin{tabu} to 0.8\textwidth { | X[l] | X[l] | p{6cm} | }
        \hline
        \multicolumn{1}{|c|}{German} & \multicolumn{1}{|c|}{English} & \multicolumn{1}{|c|}{Examples} \\
        \hline
        antworten & answer & \specialcell{l}{Antworten Sie mir! \\ Antworten Sie auf die Frage! \\ Beantworten Sie die Frage!} \\
        \hline
        danken & thank & \specialcell{l}{Ich danke dir. \\ Ich bedanke mich.} \\
        \hline
        fehlen & be missing & \specialcell{l}{Du fehlst mir \\ Was fehlt dir?} \\
        \hline
        folgen & follow & \specialcell{l}{Bitte folgen Sie mir! \\ Ich bin ihm gefolgt. \\ Ich befolge immer deinen Rat.} \\
        \hline
        gefallen & like, be pleasing to & \specialcell{l}{Dein Hemd gefällt mir. \\ Also negative, missfallen, to not like \\ Dein Hemd missfällt mir.} \\
        \hline
        gehören & belong to & Das Buch gehört mir, nicht dir. \\
        \hline
        glauben & believe & Er glaubte mir nicht. \\
        \hline
        helfen & help & \specialcell{l}{Hilf deinem Bruder! \\ Ich kann dir leider nicht helfen.} \\
        \hline
        Leid tun & be sorry & \specialcell{l}{Es tut mir Leid. \\ Sie tut mir Leid.} \\
        \hline
        passieren & to happen (to) & Was ist dir passiert? \\
        \hline
        verzeihen & pardon, forgive & Ich kann ihm nicht verzeihen. \\
        \hline
        wehtun & to hurt & Wo tut es Ihnen weh? \\
        \hline
    \end{tabu}
    \caption{Dative verbs: Commonly used}
    \label{tab:dat_verbs}
\end{table}

\begin{table}[H]
    \centering
    \begin{tabu} to 0.4\textwidth { | X[l] | X[r] | }
        \hline
        zu + dem & zum\\
        \hline
        zu + der &  zur\\
        \hline
        von + dem &  vom\\
        \hline
        bei + dem &  beim\\
        \hline
        in + das &  ins\\
        \hline
    \end{tabu}
    \caption{Abbreviations}
    \label{tab:abbreviations}
\end{table}


\section{Weird phrases and div}
Viel for uncountable/mass nouns (viel Zucker, viel Spaß, viel deutsch) 
\\ \\
$\textbf{Jede}$ at beginning of a sentence - $\textbf{Pro}$ at end
\\ \\
Ich erinnere mich an ... - I remember
\\ \\
We replace the orange with a potato - Wir  die Orange durch eine Kartoffel
\\ \\
Both - Sowohl als auch
\\ \\
As good as them - So gut wie sie
\\ \\
She never liked me either - Mich hat sie auch nie gemocht
\\ \\
Did he look like a doctor? - Sah er wie ein Arzt aus?
\\ \\
As of now - Ab jetzt
\\ \\
Over and over again - Immer wieder
\\ \\
Practice makes perfect - Übung macht den Meister
\\ \\
Something else? - Sonst noch etwas?


\section{Random words}
Ansonsten - Otherwise
\\ \\
Allerdings - However
\\ \\
Bisher - So far
\\ \\
Darum - Because
\\ \\
Deshalb - that is why
\\ \\
Dadurch - As a result/That is why
\\ \\
Dabei - With that
\\ \\
Damit - So that/with that
\\ \\
Dazu - About it
\\ \\
Gab - Was
\\ \\
Denn/Weil - Because
\\ \\
Kleider is both "clothes" and "dresses"

\end{document}
